%%%%%%%%%%%%%%%%%%%%%%%%%%%%%%%%%%%%%%%%%%%%%%%%%%%%%%%%%%%%%%%%%%
%%%%%%%%%%%%%%%%%%%%%%%%%%%%%%%%%%%%%%%%%%%%%%%%%%%%%%%%%%%%%%%%%%
%Packages
\documentclass[10pt, a4paper]{article}
\usepackage[top=3cm, bottom=4cm, left=3.5cm, right=3.5cm]{geometry}
\usepackage{amsmath,amsthm,amsfonts,amssymb,amscd, fancyhdr, color, comment, graphicx, environ}
\usepackage{float}
\usepackage{blindtext}
\usepackage{titlesec}
\usepackage{cancel}
\usepackage{mathrsfs}
\usepackage[math-style=ISO]{unicode-math}
\setmathfont{TeX Gyre Termes Math}
\usepackage{lastpage}
\usepackage[dvipsnames]{xcolor}
\usepackage[framemethod=TikZ]{mdframed}
\usepackage{enumerate}
\usepackage[shortlabels]{enumitem}
\usepackage{fancyhdr}
\usepackage{indentfirst}
\usepackage{listings}
\usepackage{sectsty}
\usepackage{thmtools}
\usepackage{shadethm}
\usepackage{hyperref}
\usepackage{setspace}
\hypersetup{
    colorlinks=true,
    linkcolor=blue,
    filecolor=magenta,      
    urlcolor=blue,
}
%%%%%%%%%%%%%%%%%%%%%%%%%%%%%%%%%%%%%%%%%%%%%%%%%%%%%%%%%%%%%%%%%%
%%%%%%%%%%%%%%%%%%%%%%%%%%%%%%%%%%%%%%%%%%%%%%%%%%%%%%%%%%%%%%%%%%
%Environment setup
\mdfsetup{skipabove=\topskip,skipbelow=\topskip}
\newrobustcmd\ExampleText{%
An \textit{inhomogeneous linear} differential equation has the form
\begin{align}
L[v ] = f,
\end{align}
where $L$ is a linear differential operator, $v$ is the dependent
variable, and $f$ is a given non−zero function of the independent
variables alone.
}
\mdfdefinestyle{theoremstyle}{%
linecolor=black,linewidth=1pt,%
frametitlerule=true,%
frametitlebackgroundcolor=gray!20,
innertopmargin=\topskip,
}
\mdtheorem[style=theoremstyle]{Problem}{Problem}
\newenvironment{Solution}{\textbf{Solution.}}

\definecolor{codegreen}{rgb}{0,0.6,0}
\definecolor{codegray}{rgb}{0.5,0.5,0.5}
\definecolor{codepurple}{rgb}{0.58,0,0.82}
\definecolor{backcolour}{rgb}{0.95,0.95,0.92}

\lstdefinestyle{mystyle}{
    backgroundcolor=\color{backcolour},   
    commentstyle=\color{codegreen},
    keywordstyle=\color{magenta},
    numberstyle=\tiny\color{codegray},
    stringstyle=\color{codepurple},
    basicstyle=\ttfamily\footnotesize,
    breakatwhitespace=false,         
    breaklines=true,                 
    captionpos=b,                    
    keepspaces=true,                 
    numbers=left,                    
    numbersep=5pt,                  
    showspaces=false,                
    showstringspaces=false,
    showtabs=false,                  
    tabsize=2
}

\lstset{style=mystyle}
%%%%%%%%%%%%%%%%%%%%%%%%%%%%%%%%%%%%%%%%%%%%%%%%%%%%%%%%%%%%%%%%%%
%%%%%%%%%%%%%%%%%%%%%%%%%%%%%%%%%%%%%%%%%%%%%%%%%%%%%%%%%%%%%%%%%%
%Fill in the appropriate information below
\newcommand{\norm}[1]{\left\lVert#1\right\rVert}     
\newcommand\course{BT5051: Transport Phenomena in Biological Systems}                            % <-- course name   
\newcommand\hwnumber{2}                                 % <-- homework number
\newcommand\Information{Sumedh Sanjay Kangne (BE21B040) \\ Department of Biotechnology \\ Indian Institute of Technology, Madras}                        % <-- personal information
%%%%%%%%%%%%%%%%%%%%%%%%%%%%%%%%%%%%%%%%%%%%%%%%%%%%%%%%%%%%%%%%%%
%%%%%%%%%%%%%%%%%%%%%%%%%%%%%%%%%%%%%%%%%%%%%%%%%%%%%%%%%%%%%%%%%%
%Page setup
\pagestyle{fancy}
\headheight 35pt
\lhead{\today}
\rhead{\includegraphics[width=0.5cm]{iitmlogo.png}}
\lfoot{}
\pagenumbering{arabic}
\cfoot{\small\thepage}
\rfoot{}
\headsep 1.2em
\renewcommand{\baselinestretch}{1.25}
%%%%%%%%%%%%%%%%%%%%%%%%%%%%%%%%%%%%%%%%%%%%%%%%%%%%%%%%%%%%%%%%%%
%%%%%%%%%%%%%%%%%%%%%%%%%%%%%%%%%%%%%%%%%%%%%%%%%%%%%%%%%%%%%%%%%%
%Add new commands here
\renewcommand{\labelenumi}{\alph{enumi})}
\newcommand{\Z}{\mathbb Z}
\newcommand{\R}{\mathbb R}
\newcommand{\Q}{\mathbb Q}
\newcommand{\NN}{\mathbb N}
\newcommand{\PP}{\mathbb P}
\DeclareMathOperator{\Mod}{Mod} 
\renewcommand\lstlistingname{Algorithm}
\renewcommand\lstlistlistingname{Algorithms}
\def\lstlistingautorefname{Alg.}
\newtheorem*{theorem}{Theorem}
\newtheorem*{lemma}{Lemma}
\newtheorem{case}{Case}
\newcommand{\assign}{:=}
\newcommand{\infixiff}{\text{ iff }}
\newcommand{\nobracket}{}
\newcommand{\backassign}{=:}
\newcommand{\tmmathbf}[1]{\ensuremath{\boldsymbol{#1}}}
\newcommand{\tmop}[1]{\ensuremath{\operatorname{#1}}}
\newcommand{\tmtextbf}[1]{\text{{\bfseries{#1}}}}
\newcommand{\tmtextit}[1]{\text{{\itshape{#1}}}}

\newenvironment{itemizedot}{\begin{itemize} \renewcommand{\labelitemi}{$\bullet$}\renewcommand{\labelitemii}{$\bullet$}\renewcommand{\labelitemiii}{$\bullet$}\renewcommand{\labelitemiv}{$\bullet$}}{\end{itemize}}
\catcode`\<=\active \def<{
\fontencoding{T1}\selectfont\symbol{60}\fontencoding{\encodingdefault}}
\catcode`\>=\active \def>{
\fontencoding{T1}\selectfont\symbol{62}\fontencoding{\encodingdefault}}
\catcode`\<=\active \def<{
\fontencoding{T1}\selectfont\symbol{60}\fontencoding{\encodingdefault}}

%%%%%%%%%%%%%%%%%%%%%%%%%%%%%%%%%%%%%%%%%%%%%%%%%%%%%%%%%%%%%%%%%%
%%%%%%%%%%%%%%%%%%%%%%%%%%%%%%%%%%%%%%%%%%%%%%%%%%%%%%%%%%%%%%%%%%
%Begin now!



\begin{document}

\begin{titlepage}
    \begin{center}
        \vspace*{3cm}
            
        \Huge
        \textbf{BT5051: Transport Phenomena in Biological Systems}
            
        \vspace{1cm}
        \huge
        Ethylene Diffusion in Fruits \\
        \Large
        CFA: Choose Focus Analyse
            
        \vspace{1.5cm}
        \Large
            
        \textbf{\Information}                      % <-- author
        
            
        \vfill
        
        A \course \ Assignment
            
        \vspace{1cm}
            
        \includegraphics[width=0.2\textwidth]{iitmlogo.png}
        \\
        
        \Large
        
        \today
            
    \end{center}
\end{titlepage}

%%%%%%%%%%%%%%%%%%%%%%%%%%%%%%%%%%%%%%%%%%%%%%%%%%%%%%%%%%%%%%%%%%
%%%%%%%%%%%%%%%%%%%%%%%%%%%%%%%%%%%%%%%%%%%%%%%%%%%%%%%%%%%%%%%%%%
%Start the assignment now
%%%%%%%%%%%%%%%%%%%%%%%%%%%%%%%%%%%%%%%%%%%%%%%%%%%%%%%%%%%%%%%%%%
%New problem
\newpage
\huge

\begin{center}
    Preface\\
    \\
    \large
    This exercise attempts to model the ripening of fruit as a cause of ethylene diffusion. All assumptions made are listed in the report. All references used are also cited accordingly.\\
    I express my gratitude to Professor G.K. Suraishkumar for his exemplary implementation of diverse teaching methodologies throughout this course, allowing me the opportunity to engage in this exercise. I am immensely appreciative of his innovative teaching approaches, which have been instrumental in shaping my learning experience.\\
    I also am deeply grateful to those who have supported, guided, and inspired me throughout this endeavor. Their contributions, whether profound or subtle, have been instrumental in shaping the essence of this creation.
\end{center}

\newpage
\Large
\tableofcontents

\newpage
\huge 
\section{Introduction} \\
\large
\subsection{What is a fruit?}
Fruits have been a part of the human diet throughout evolution. Year after year, season after season people have enjoyed this sweet delicacy in many forms yet only a few question its significance. Let’s look at this from a plant’s perspective. \\
For a plant, its main objective is to reproduce and keep the species alive. Plant reproduction is the process of producing new offspring either sexually or asexually. Asexual reproduction doesn’t involve any zygote formation whereas sexual reproduction leads to gamete fusion in the process of fertilisation resulting in a zygote which further forms the embryo part of the seed. \cite{Copeland1999} \\
Dispersal of seed is a very important part of seed germination. It allows diversification and colonization of various suitable habitats. This spread of seeds further reduces the competition for nutrition between parent-offspring or among offspring themselves. Frugivorous Animals, a.k.a animals that thrive on a full-fledged fruit-based diet, play a very important role in the dispersion of seeds. They eat these fruits and then proceed to disperse the seeds through their faecal matter. Therefore, seeds that are still quite dormant and lack the stuff they need to sprout when gobbled down by these animals, fail to germinate on dispersion. \\
This is where fruit comes into the picture. So, when the seed is still immature the fruit has a very thick skin, unappealing color, hard flesh, and bitter chemical compounds making the fruit undesirable. This is what we call a ‘Raw Fruit’. Once the seed is ready, the fruit starts changing its appearance. The skin starts to thin, the color becomes more saturated with pigment, and the flesh turns into a sweet pulpy goodness. Papaya (Carica papaya) when exogenously treated with Ethylene ($100 \mu l l^{-1}$) shows increased carotenoid production and a decrease in chlorophyll concentrations as well as softening of the fruit’s flesh. \cite{an1990storage}. The fruit now has become an irresistible treat for these animals and the seed is now ready for a new journey ahead. \\

\subsection{Ripening of a fruit}
Ripening of a fruit is an irreversible process comprising of many biochemical reactions resulting in the above-mentioned physiological and organoleptic changes \cite{article}. The hard starchy flesh of the fruit, bananas for example, starts breaking down and after many metabolic pathways \cite{seymour2012biochemistry} the starch content gets converted to sugars such as glucose, fructose or sucrose, imparting that familiar soft and sweet taste to the fruit. Similar reactions are seen in the case of color changes. Many fruits show degradation of chlorophyll which thereby increases Carotene concentrations making the fruit appear orange-red in color \cite{seymour2012biochemistry}. All these changes are brought by some kind of ripening-inducing hormone such as ethylene gas in most cases. Ethylene, a naturally produced plant growth hormone, exerts diverse impacts on the growth, maturation, and shelf life of various fruits even at low concentrations of micro l l-1 \cite{SALTVEIT1999279}. Ethylene was the first ever gaseous hormone to be discovered and hence studied for decades. The action and biosynthesis of ethylene \cite{adams1979ethylene} coherently relate to the process of respiration and need oxygen as a substrate and release Carbon dioxide as a product. We can thus see a direct relation between ripening and respiration \cite{tripathi2016fruit}. 
\begin{figure}[H]
    \centering
    \includegraphics[scale=1]{Cross-section-passing-through-a-fruit-showing-how-the-concentration-of-O-2-and-CO-2-can.png}
    \caption{Variation in concentration levels of Oxygen and Carbon dioxide gas}
\end{figure}
 \emph{Figure 1} showcases the variation in the concentration of Oxygen and Carbon dioxide in the mesocarp region of the fruit. The concentration of oxygen seems to be decreasing as we go deeper inside the fruit, whereas the opposite is seen in the case of carbon dioxide \cite{inbook}. Therefore, the outermost layer of the mesocarp has the highest Oxygen concentration and lowest Carbon dioxide concentration, both favouring the respiration process and thereby favouring ripening. Hence, we can say a fruit ripens from the outside and all the ethylene generated inside the fruit tends to freely diffuse through cells \cite{alexander2002ethylene} in an outward direction.

\subsection{Types of fruit (Based on respiration levels during ripening)}
Have you ever bought fruits that weren’t fully ripened and kept them on your countertop or your fruit basket and in a few days they magically turned ripe, ready to be eaten. Although, this is not the case with all fruits. Fruits like pineapples, grapes, oranges, etc. tend to import ripening hormones and sugars from other parts of the plants, but once they are separated from the plant or harvested per se, ripening stops. Hence, oranges or pineapples won’t ripen just by sitting on your countertop and thus need to be fully ripened before harvesting. These fruits are called Non-Climacteric Fruits. Other fruits like Apples, Bananas, Tomatoes, etc. which continue to ripen even after being picked from plants are called Climacteric Fruits.
\begin{figure}[H]
    \centering
    \includegraphics[scale=0.65]{Table.png}
    \caption{Classification of fruits into climacteric and non-climacteric groups}
\end{figure}
\cite{paul2012fading} \\

Another important distinction between Climacteric and Non-climacteric fruits is the difference in their rate of respiration with respect to increasing ethylene concentrations. Climacteric fruits show an increased rate of respiration whereas in the case of Non-Climacteric fruits, no significant change is seen. This also validates how non-climacteric fruits tend to stop their ripening process once plucked off the plant.
\begin{figure}[H]
    \centering
    \includegraphics[scale=5.3]{Climacteric Non-Climacteric distinction.jpg}
    \caption{Variation of Respiration levels with increasing ethylene concentrations}
\end{figure}

\newpage
\section{Background}
As we have seen earlier, an unripe climacteric fruit responds to increasing ethylene concentrations and starts ripening. Now let’s try to figure out how this works and what tools we have at hand.

\subsection{Principle of Mass Conservation}
‘Mass can neither be created, nor destroyed’, this principle doesn’t hold when considering nuclear reaction. However, we won’t be seeing any nuclear reactions in our case and therefore we can overlook this exception. \\

Continuum, a macroscopic situation where a large number of molecules are present per unit volume of some substance.  Hence the system can be considered to be continuous. Our system is of a macroscopic scale and thus continuum can be safely assumed. \\

Considering a system, any component in that system can only undergo any of these 5 things: Enter the system (I), Exit the system (O), can be formed in the system (G), can be utilised in the system (C), and can accumulate in a system (A). 

\[I - O + G - C = A\]
\\
Writing in terms of rates, \\

\begin{equation}
    \dot{I} - \dot{O} + \dot{G} - \dot{C} = \frac{dA}{dt}
\end{equation}
\\
Where,\\
\\
$\dot{I}$ is the Rate of input of A into the system \\
$\dot{O}$ is the Rate of output of A from the system \\
$\dot{G}$ is the Rate of generation of A in the system \\
$\dot{C}$ is the Rate of consumption of A in the system \\
$\frac{dA}{dt}$ is the Rate of accumulation of A in the system \\

\subsection{Fick’s First Law}
Molar flux or movement of molecules will be in the negative direction of the acting driving force i.e. from a higher concentration to a lower concentration.\\

\begin{equation}
    \bigg. J_{i}^* = -D_{i} \frac{\partial c_{i}}{\partial x}
\end{equation} \\
\\
where,\\
\\
$J_{i}^*$ is the Molar flux of the $i^{th}$ component\\
$D_{i}$ is the Diffusivity coefficient of the $i^{th}$ component\\
$c_i$ is the concentration of the $i^{th}$ component\\

\subsection{Fick’s Second Law}
Gives a direct relation between the rate of accumulation and both the diffusivity and the second-order derivative of concentration.\\

\begin{equation}
    \frac{\partial c_{i}}{\partial t} = D_{i} \frac{\partial^{2} c_{i}}{\partial x^2} + R_{i}
\end{equation}\\
\\
Here, $R_{i}$ is the rate of generation of the $i^{th}$ component\\

\subsection{Unsteady state}
The term "unsteady state" refers to a condition or situation where a system or process is not in a stable, constant, or equilibrium state. Instead, it is in a state of change, transition, or dynamic evolution. This could involve variations, fluctuations, or alterations in the system's parameters, characteristics, or behavior over time.  Accumulation or change in concentration with respect to time is an observable effect of  unsteady state.

\subsection{Pseudo steady state}
Pseudo steady state is a concept often used in the context of chemical reactions or other dynamic systems to describe a condition where,the rates of change of certain variables within the system may be so slow or occur at different time scales that, for a period, the system seems to reach a quasi-equilibrium condition. 


\section{Model}
We have gone over all the tools we will need and can now start modeling the diffusion of ethylene in  McIntosh apples, which are our choice of apples for this model. They are small-medium sized apples found in the new york region.\\
Another point to consider is that we will be trying to model the process of ripening from the harvest of the fruit till it is perfectly suitable for consumption.\\  
Let’s begin by considering the apple as a perfect sphere. This will help us simplify the problem and use various equations without affecting the result. \\
Let’s begin by considering the apple as a perfect sphere. This will help us simplify the problem and use various equations without affecting the result. \\

\begin{figure}[H]
    \centering
    \includegraphics[scale=1]{AppleCrossSection.png}
    \caption{Phase Plot of the system when $I_{ext}$ = 0}
\end{figure} \\

In figure x we have taken an element of thickness dr. Let’s try to write the mass balance equation for this element \\

\begin{equation}
    \dot{I} - \dot{O} + \dot{G} - \dot{C} = \frac{dA}{dt}
\end{equation}
\\
Now we know that ethylene is a hormone and thus won't be consumed in a chemical reaction, but there is a loss of ethylene from the fruit to the surroundings. However, This loss is so less \cite{SALTVEIT1999279} that we can assume pseudo steady state and consider it as zero. Therefore, \\

\begin{equation}
    \dot{I} - \dot{O} + \dot{G} - \cancel{\dot{C}} = \frac{dA}{dt}
\end{equation}
\\
Now let’s use the constitutive equation for spherical coordinate system,\\
\begin{equation}
    \frac{\partial c_i}{\partial t} + (v_r\frac{\partial c_i}{\partial r} + v_{\theta}\frac{1}{r} \frac{\partial c_i}{\partial \theta} + v_{\phi}\frac{1}{rsin\theta } \frac{\partial c_i}{\partial \phi }) - D_i(\frac{1}{r^2}\frac{\partial}{\partial r}(r^2\frac{\partial c_i}{\partial r}) + \frac{1}{r^2sin\theta }\frac{\partial}{\partial \theta}(sin\theta\frac{\partial c_i}{\partial \theta }) + \frac{1}{r^2sin^2\theta }\frac{\partial^2c_i}{\partial \phi^2}) = R_i
\end{equation} \\
Since there is no bulk flow all the velocity terms will be zero. Also since concentration does not vary along the theta and phi directions, therefore their derivatives will be zero. \\
\begin{equation}
    \frac{\partial c_i}{\partial t} + \cancelto{0}{(v_r\frac{\partial c_i}{\partial r} + v_{\theta}\frac{1}{r} \frac{\partial c_i}{\partial \theta} + v_{\phi}\frac{1}{rsin\theta } \frac{\partial c_i}{\partial \phi })} - D_i(\frac{1}{r^2}\frac{\partial}{\partial r}(r^2\frac{\partial c_i}{\partial r}) + \cancelto{0}{\frac{1}{r^2sin\theta }\frac{\partial}{\partial \theta}(sin\theta\frac{\partial c_i}{\partial \theta })} + \cancelto{0}{\frac{1}{r^2sin^2\theta }\frac{\partial^2c_i}{\partial \phi^2})} = R_i
\end{equation}\\
\\
Therefore, the equation now reduces to\\

\begin{equation}
    \frac{\partial c_i}{\partial t} - D_i(\frac{1}{r^2}\frac{\partial}{\partial r}(r^2\frac{\partial c_i}{\partial r})) = R_i
\end{equation}\\
\\
Now since both rate of accumulation and rate of generation are only dependent on time we can consider them as a single term. \\

\begin{equation}
    \label{differential}
    \frac{\partial C_i}{\partial t} = D_i(\frac{1}{r^2}\frac{\partial}{\partial r}(r^2\frac{\partial c_i}{\partial r}))
\end{equation}\\
\\
Where $\frac{\partial C_i}{\partial t}$ is the change in concentration of $i^{th}$ component due to both accumulation and generation.\\

Let's look at the boundary conditions: \\
\[BC\;\; 1:\quad t=0\quad\quad\quad 0\leq r\leq R\quad\quad\quad C_i=0\]
\[BC\;\; 2:\quad t\geq0\quad\quad\quad r=0 \quad\quad\quad\quad \frac{\partial C_i}{\partial r}=0\]
\[BC\;\; 3:\quad t\geq0\quad\quad\quad r=R \quad\quad\quad\quad C_i=C_1\]\\
\\
Now, let us define some dimensionless variables:\\

\[\theta=\frac{C_i}{C_1},\]\\
\[\eta=\frac{r}{R},\]\\
\[\tau=\frac{tD_i}{R^2}\]\\
\\
Putting the above-mentioned dimensionless variables in \emph{Eq. 9}, we get\\

\begin{equation}
    \frac{\partial \theta}{\partial \tau} = \frac{1}{\eta^2}\frac{\partial}{\partial \eta}(\eta^2\frac{\partial \theta}{\partial \eta}))
\end{equation}\\
\\
Now, the boundary conditions will be\\
\[BC\;\; 1:\quad \tau=0\quad\quad\quad 0\leq \eta\leq 1\quad\quad\quad \theta=0\]
\[BC\;\; 2:\quad \tau\geq0\quad\quad\quad \eta=0 \quad\quad\quad\quad \frac{\partial \theta}{\partial \eta}=0\]
\[BC\;\; 3:\quad \tau\geq0\quad\quad\quad\quad \eta=1 \quad\quad\quad\quad \theta=1\]\\
\\
Using the following figure,
\begin{figure}[H]
    \centering
    \includegraphics[scale=0.4]{phpAJPehW.png}
    \caption{Normalised concentration vs Dimensionless Position for different Dimensionless time values}
\end{figure} \\
\\
\newpage
\section{Analysis}
\subsection{Results}
Looking at \emph{Fig. 5} we can clearly make an inference that the concentration level at r=R are maximum, thereby validating our claim of fruit ripening in an inwards direction from the outer layer of mesocarp. \\
\\
Further, we know\\
\[\tau=\frac{tD_i}{R^2}\]\\
therefore,\\

\[t=\frac{\tau R^2}{D_i}\]\\

Here, R can be taken as $3.5cm$ \cite{McIntosh} and the diffusivity coefficient for ethylene in apples is in the order of $10^{-5} cm s^{-1}$ \cite{diffusivityCoeff}.\\
In \emph{Fig. 5} when the concentration curve flattens out at $\tau=0.5$ is the time when steady state is achieved. Therefore, \\

\[t=\frac{0.5 (3.5cm)^2}{10^{-5}cm s^{-1}}\]\\
\[t=\frac{6.125}{10^{-5}} s\]\\
\[t=612500s\]\\
\[t\approx7days\]\\
\\
This result shows that it takes roughly 7 days for an apple (McIntosh in this case) to ripen after being harvested. This means there is roughly a 7-day span where the apple will remain fresh before it starts over-softening and senescence \cite{WSU}

\subsection{Assumptions}
Listed below are the assumptions made while working out this problem:
\begin{enumerate}
    \item Considering apples as a perfect sphere
    \item Initial concentration was considered zero
    \item The time frame considered is that of post-harvest apples.
\end{enumerate}

\newpage
\section{Industrial Impact}
The above-mentioned results can help predict many logistical issues a farmer might face while cultivating apples or any fruit or vegetable for that matter. Let's look at the impact each of these results might exhibit on the industry. \\
\begin{enumerate}
    \item Fruit ripens from the outside to the core. This can allow farmers to identify the beginning of the actual ripening process and therefore indicate that it's time for harvest.
    \item Using the model, we can figure out the time for which the fruit keeps on ripening without going bad. This allows us to properly arrange its timeline of harvest, transport, delivery, and then further keep track of its shelf life when it's just sitting in the supermarket.
\end{enumerate}

\bibliographystyle{plain}
\bibliography{AllBib}
%%%%%%%%%%%%%%%%%%%%%%%%%%%%%%%%%%%%%%%%%%%%%%%%%%%%%%%%%%%%%%%%%%
%Complete the assignment now
\end{document}

%%%%%%%%%%%%%%%%%%%%%%%%%%%%%%%%%%%%%%%%%%%%%%%%%%%%%%%%%%%%%%%%%%
%%%%%%%%%%%%%%%%%%%%%%%%%%%%%%%%%%%%%%%%%%%%%%%%%%%%%%%%%%%%%%%%%%
